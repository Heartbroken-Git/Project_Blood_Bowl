\documentclass{article}
\usepackage[utf8]{inputenc}
\usepackage[T1]{fontenc}
\usepackage{lmodern}
\usepackage{graphicx}
\usepackage[frenchb]{babel}
\usepackage{hyperref}
\usepackage[table,xcdraw]{xcolor}
\usepackage{float}

\newcommand{\info}{\texttt}
\newcommand{\pattern}{\emph}
\newcommand{\ue}{\textbf{X5I0010 "Objet et développement d'applications"}}

\title{Objet et développement d'applications - Design Patterns\\
Projet libre : BloodBowl}
\author{Elbert Noel \bsc{NYUNTING} \and Corentin \bsc{CHÉDOTAL}}
\date{19 Novembre 2016}

%PEUT ETRE CHANGER LES MARGES DU DOCUMENT POUR AVOIR PLUS DE PLACE

\begin{document}

\maketitle

\centerline{\includegraphics[scale=0.75]{img/logo.png}}

\begin{abstract}
    Dans le cadre de l'UE X5I0010 "Objet et développement d'applications" nous avons réalisé ce projet. Il comporte comme demandé plusieurs "designs patterns". Ont été utilisé les patterns \pattern{Strategy}, \pattern{State}, \pattern{Command}, \pattern{Abstract Factory}. Ce rapport expliquera leur implémentation, le but et fonctionnement du projet ainsi que les difficultés rencontrées dans la réalisation de celui-ci.
\end{abstract}

\newpage

\tableofcontents

\newpage

\section{Introduction}
    
    L'Unité d'Enseignement \ue requérait la réalisation d'un projet de fin de semestre. Nous avions carte blanche dans le choix du sujet du projet (dans les limites du raisonnable bien sur). Notre binôme ayant un fort intérêt pour les jeux vidéos nous avons très tôt décidé de réaliser un prototype d'un jeu. Nous sommes partis sur de la stratégie au tour par tour car c'était le genre qui nous paraissait le plus réalisable dans le temps imparti tout en étant très ouvert aux différents patterns vus en cours. Enfin nous nous sommes raccrochés à une série de règles familière (BloodBowl) afin de ne pas perdre trop de temps dans le Game Design. Les règles ont cependant été évidemment adaptée au fonctionnement du programme\footnote{Les règles originales ainsi qu'une aide de jeu sont mises à disposition dans \info{/docs}} et pour permettre une inclusion plus facile des patterns.\\
    Très tôt dans la réalisation du projet il a été décidé de ne pas réaliser une interface graphique particulière. Toutes les interactions avec l'Utilisateur se faisant par le biais du terminal, à nouveau dans un but de gain de temps.

\section{Le projet}
    
    %DESCIRPTION SPECIFIQUE DU PROJET
    
    \subsection{Description du projet et de son fonctionnement}
    %INSERER DIAGRAMME COMPLET

    \subsection{Principales classes}

        \subsubsection{Classe Poney}

    \subsection{Design patterns employés}

        \subsubsection{Design \pattern{Strategy}}
        %INSERER DIAGRAMME
        
\section{Commentaires sur le projet}

    \subsection{La licence employée}
    
    %PAS OBLIGATOIRE
    
    \subsection{Difficultés rencontrées}
    
    Le projet a fait face à quelques sévères difficultés durant son développement. Que ce soit des erreurs qui auraient pu être catastrophiques, la mise en place de nouveaux systèmes pas encore bien compris ou le temps passant très vite, les problèmes les plus importants que nous avons rencontrés sont explicités ci-dessous.
    
        \subsubsection{La \emph{Pushocalypse}}
        
        \info{Git} est un système de gestion de version particulièrement efficace et pratique. Nous l'avons utilisé dès la genèse du projet. Cependant tout produit puisant vient avec son lot de pièges et embûches. L'une d'entre elle la capacité de \info{git push -f} d'écraser l'arborescence du répertoire distant et de la remplacer par celle locale. Fonction qui peut être très pratique pour résoudre certains problèmes mais qui causa la suppression des modifications que l'un avait fait alors que l'autre fit un \info{git push -f} par mauvaise habitude alors qu'il n'avait pas récupéré les derniers commits en date avant. Fort heureusement une copie locale pré-\emph{Pushocalypse} put être récupérée et seulement deux petits commits durent être refait.
        
        \subsubsection{\info{Travis CI} ou comment je me suis battu pour mettre en place une belle-mère virtuelle}
        
        Afin de faciliter le débuggage et les test durant le développement surtout quand nous n'étions pas à travailler côte à côte. Il avait été décidé de mettre en place une integration \info{Travis CI} très tôt dans le répertoire GitHub. Celui permettant alors d'avoir des logs précis, l'exécution de tests à chaque push sur la branche principale et le contrôle continu par \info{valgrind} contre d'éventuelles fuites mémoires.\\
        Seulement, aucun d'entre nous n'avait usé de \info{Travis} avant, or la réalisation du script \info{.travis.yml} ne fut pas la plus aisée. Elle prit une semaine par Corentin de bien mettre en place.
        
        \subsubsection{Débuggage des classes Action et Tile, la belle-mère contre attaque}
        
        La classe Action et ses sous-classes\dots Un élément primordial et principal de notre programme. Étant une classe qui gère les actions des joueurs des équipes, en utilisant la classe Ball et Dice, Elbert à très mal codé les classes pour commencer. Il s'est revendiquer en passant une semaine sur \info{Travis} (ou la belle mère) et débuggait les liens entre les classes différentes et fautes de frappes. Ensuite Elbert a débuggé la classe Tile, qui contient potentiellement un Player en attribut. N'étant pas sur Java, nous ne pouvons pas mettre un objet en NULL (cas où il n'y a pas je Player sur le Tile). Elbert a déduit qu'un pointeur est nécessaire, avec un Tile contenant un pointeur de Player. Il a ensuite déduit qu'il fallait utiliser un smart pointer... Un \info{unique\_ptr} . Après toute une journée avec la belle mère qui affichait des erreurs de compilations inconnues (ni par Elbert, ni par Corentin, ni par les M1 et M2 ALMA) Elbert s'est rendu compte que \info{unique\_ptr} ne peut pas être égal à zéro. C'est pour cela que nous avons décidé d'utiliser \info{shared\_ptr} pour contenir les pointeurs de Player dans Tile. Une semaine et demi, près de 300 commits, et au moins une quarantaine cafés noisettes plus tard, la belle mère affichait enfin zéro erreurs et nous laissait enfin de mettre les classes de Action et Tile dans la branche master de nôtre git.
        
        \subsubsection{\bsc{LAMARTINE} serait fier de nous}
        
        Ce projet fut aussi l'occasion de nous rappeler combien le poète romantique avait raison. Le temps coule et même si nous cherchons à le remonter c'est impossible. En effet bien que très bien parti initialement nous avons très vite été rattrapé par les autres projets des autres UE et il est vite devenu difficile de gérer son temps. Nous avons donc fini acculé par celui-ci. Bien que nous n'avons pas l'impression d'avoir bâclé la réalisation du programme, il serait présomptueux que de dire que nous n'avons pas passé quelques longues nuits sur le projet afin de finir dans les temps. 
    
    \subsection{Autres}

\end{document}
